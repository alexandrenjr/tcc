\chapter{Introdução}
A Teoria de Controle desenvolveu e utilizou várias técnicas de projetos de compensadores para vários tipos de sistemas físicos. Uma área de extrema importância é o controle digital, onde através de sensores, leem as entradas ou saídas em períodos de tempo especificados. Apesar de ser relativamente recente, há extensa literatura disponível.

Em relação ao seu dual, os sistemas contínuos, existem técnicas via Desigualdades Matriciais Lineares que alocam polos dentro de uma região específica. Tal feito é realizado devido à convexidade das regiões de interesse, o que não ocorre para os sistemas discretos.

Nos últimos anos, houveram estudos que aproximam tais regiões via LMIs em regiões complexas, como em \citeonline {ROSINOVA2019} e \citeonline{WISNIEWSKI2017}, através de regiões simples como setores cônicos, semi-planos e elipses.

Contudo, a união de vários tipos de regiões em um único algoritmo se torna necessário para projetos simples de compensadores referentes a sistemas discretos. Logo, este trabalho reuniu tais teorias a fim de desenvolver em \emph{software} tais regiões e verificar a factibilidade da solução.

Notação:
\begin{itemize}
\item $\Herm{(A)} = A + A'$, onde $A$ é uma matriz simétrica e $A'$ a sua transposta;
\item $\herm{(A)} = A - A'$, onde $A$ é uma matriz simétrica e $A'$ a sua transposta;
\item $\loc(v_1, v_2)$ determina o ponto que a reta que passa por $v_1$ e $v_2$ cruza o eixo real;
\item $\angulo(v_1, v_2)$ refere-se ao ângulo entre aquela reta e o eixo real.
\end{itemize}