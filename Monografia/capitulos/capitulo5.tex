\chapter{Conclusão}
A proposta de unir diferentes aproximações para regiões diferentes em uum único algoritmo foi cumprida. As principais dificuldades encontradas foram verificar a fatibilidade da solução, uma vez que solucionadores númericos podem aceitar infactibilidades no conjunto. Dessa maneira, o projeto que venha usar o algoritmo proposto deve ser conservador. Contudo, foi implementado a opção de verificar a solução proposta pelo interpretador, caso não seja factível. Caso os resíduos sejam ínfimos (na ordem de $10^{-20}$), é possível utilizar tal solução. Cabe ao usuário aceitar ou não a infactibilidade, nesses casos.

Ainda, como a resolução de LMIs foi feita utilizando programação semi-definida, a presença de uma função objetiva tornou dificultoso vários projetos. A função proposta foi o traço de $P$, onde minimiza os auto-valores da matriz de estado. Contudo, como a ideia do algoritmo é determinar um ponto de factibilidade, optou-se por não incluir a função objetiva. Mesmo assim, as soluções propostas, caso atendem ao parâmetros de projeto, se mostraram corretas.

Em relação às aproximações estudadas, foi possível notar que as cônicas, por serem muito simples, permitem soluções infactíveis. No caso do exemplo apresentado, foi possível tornar o sistema $\mathscr{D}$-estável, apesar do interpretador retornar uma negativa. Logo, apesar de simples e rápida, tal método é limitado por não aproveitar bem as regiões de interesse.

Já a aproximação elíptica se mostrou uma alternativa poderosa. As áreas são bem aproveitadas, o que reflete num projeto mais relaxado em relação à aproximação anterior. Como comentado no estudo \citeonline{CHIQUETO2021}, a grande dificuldade deste método está na computação da LMI correspondente.

Finalmente, as aproximações poligonais são as que melhor aproveitam as áreas de interesse. Em contrapartida, são mais lentos. Para projetos restritivos, podem levar segundos papra retornar uma solução, devido à condição de desistência comentada na seção \ref{sec:Algoritmo}

Para trabalhos futuros, pode ser implementada a minimização da norma $H_{\infty}$, pois há literatura disponível e como há computação nas dimensões das matrizes na representação via espaço de estados, é possível incluir tal restrição no algoritmo (tarefa não realizada devido ao prazo de entrega).