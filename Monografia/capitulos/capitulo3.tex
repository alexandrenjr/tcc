\chapter{Algoritmo}

O algoritmo desenvolvido neste trabalho é um compilado de algoritmos desenvolvidos em \cite{WISNIEWSKI2017}, \cite{WISNIEWSKI2019} e \cite{CHIQUETO2021}, onde mapeiam regiões de interesse da Teoria de Controle que em alguns casos são não-convexas, em regiões aproximadamente convexas.

Em \cite{WISNIEWSKI2017} e \cite{CHIQUETO2021} foram desenvolvidos aproximações cônicas das regiões $\zeta$-constante e $\omega_n$-constante, respectivamente, via LMIS. Ainda, adaptando-se as ideias apresentadas em \cite{ROSINOVA2014} e \cite{ROSINOVA2019} para a aproximação elíptica das curvas $\zeta$-constante, em \cite{CHIQUETO2021} foi desenvolvido um algoritmo para aproximar a curva $\omega_n$-constante utlizando a maior elipse dentro da curva. Finalmente, Em \cite{WISNIEWSKI2019} e \cite{CHIQUETO2021} foram desenvolvidos aproximações poligonares para as curvas $\zeta$-constante e $\omega_n$-constante, onde aproveitam melhor tais regiões.

Referente a este trabalho, foi desenvolvido um algoritmo que baseia-se de tais ideias para sintetizar um controlador a partir de valores de $\sigma$, $\zeta$, $\omega_n$ e $T_s$ fornecidos e determinar a factibilidade da solução encontrada.

\section{Aproximação cônica}

Para o mapeamento cônico das curvas $\zeta$-constante e $\omega_n$-constante, são utilizados setores cônicos determinados via \eqref{eq:LMIESetorConicoDireito} e \eqref{eq:LMIESetorConicoEsquerdo}, e retas verticais como apresentado em \cite{WISNIEWSKI2017} e \cite{CHIQUETO2021}.

Para a primeira curva, a ideia consiste em calcular os pontos em que a espiral logarítmica cruza o eixo real, determinar o sobre o ramo o ponto entre os calculados anteriormente e partir destes, determinar as LMIs correspondentes. A figura

\begin{figure}
  \centering
  % This file was created by matlab2tikz.
%
%The latest updates can be retrieved from
%  http://www.mathworks.com/matlabcentral/fileexchange/22022-matlab2tikz-matlab2tikz
%where you can also make suggestions and rate matlab2tikz.
%
%
\begin{tikzpicture}[scale=0.65]

\begin{axis}[%
  axis lines=center,
  width=3.5in,
  height=3.5in,
  scale only axis,
  xmin=-1.2,
  xmax=1.2,
  ymin=-1.2,
  ymax=1.2,
  xtick={1},
  ytick=\empty,
  xticklabel style={anchor=north west},
  xlabel={$\pmb{X}$},
  ylabel={$\pmb{jY}$},
  x label style={anchor=north}
]
\addplot [color=black, forget plot]
  table[row sep=crcr]{%
0	1\\
0.0634239196565645	0.997986676471884\\
0.126592453573749	0.991954812830795\\
0.18925124436041	0.981928697262707\\
0.251147987181079	0.967948701396356\\
0.312033445698487	0.950071117740945\\
0.371662455660328	0.928367933016073\\
0.429794912089172	0.902926538286621\\
0.486196736100469	0.873849377069785\\
0.540640817455598	0.841253532831181\\
0.59290792905464	0.805270257531059\\
0.642787609686539	0.766044443118978\\
0.690079011482112	0.72373403810507\\
0.734591708657533	0.678509411557132\\
0.776146464291757	0.630552667084523\\
0.814575952050336	0.580056909571198\\
0.849725429949514	0.527225467610502\\
0.881453363447582	0.472271074772683\\
0.909631995354518	0.415415013001886\\
0.934147860265107	0.356886221591872\\
0.954902241444074	0.296920375328275\\
0.971811568323542	0.235758935509427\\
0.984807753012208	0.173648177666931\\
0.993838464461254	0.110838199901011\\
0.998867339183008	0.0475819158237424\\
0.999874127673875	-0.015865963834808\\
0.996854775951942	-0.0792499568567885\\
0.989821441880933	-0.142314838273285\\
0.978802446214779	-0.204806668065191\\
0.963842158559942	-0.266473813690035\\
0.945000818714669	-0.327067963317421\\
0.922354294104581	-0.386345125693128\\
0.895993774291336	-0.444066612605774\\
0.866025403784439	-0.5\\
0.832569854634771	-0.55392006386611\\
0.795761840530832	-0.605609687137666\\
0.755749574354258	-0.654860733945285\\
0.712694171378863	-0.701474887706321\\
0.666769000516292	-0.745264449675755\\
0.618158986220605	-0.786053094742787\\
0.567059863862771	-0.823676581429833\\
0.513677391573407	-0.857983413234977\\
0.458226521727411	-0.888835448654923\\
0.400930535406614	-0.916108457432069\\
0.342020143325669	-0.939692620785908\\
0.28173255684143	-0.959492973614497\\
0.220310532786541	-0.975429786885407\\
0.15800139597335	-0.987438888676394\\
0.0950560433041829	-0.995471922573085\\
0.0317279334980681	-0.999496542383185\\
-0.0317279334980679	-0.999496542383185\\
-0.0950560433041826	-0.995471922573085\\
-0.15800139597335	-0.987438888676394\\
-0.220310532786541	-0.975429786885407\\
-0.281732556841429	-0.959492973614497\\
-0.342020143325669	-0.939692620785908\\
-0.400930535406613	-0.91610845743207\\
-0.45822652172741	-0.888835448654924\\
-0.513677391573406	-0.857983413234977\\
-0.567059863862771	-0.823676581429833\\
-0.618158986220605	-0.786053094742788\\
-0.666769000516292	-0.745264449675755\\
-0.712694171378863	-0.701474887706322\\
-0.755749574354258	-0.654860733945285\\
-0.795761840530832	-0.605609687137667\\
-0.832569854634771	-0.55392006386611\\
-0.866025403784438	-0.5\\
-0.895993774291336	-0.444066612605774\\
-0.922354294104581	-0.386345125693129\\
-0.945000818714668	-0.327067963317422\\
-0.963842158559942	-0.266473813690035\\
-0.978802446214779	-0.204806668065191\\
-0.989821441880933	-0.142314838273285\\
-0.996854775951942	-0.0792499568567888\\
-0.999874127673875	-0.0158659638348076\\
-0.998867339183008	0.0475819158237424\\
-0.993838464461254	0.110838199901011\\
-0.984807753012208	0.17364817766693\\
-0.971811568323542	0.235758935509427\\
-0.954902241444074	0.296920375328275\\
-0.934147860265107	0.356886221591872\\
-0.909631995354519	0.415415013001886\\
-0.881453363447582	0.472271074772682\\
-0.849725429949514	0.527225467610502\\
-0.814575952050336	0.580056909571198\\
-0.776146464291757	0.630552667084522\\
-0.734591708657534	0.678509411557132\\
-0.690079011482113	0.723734038105069\\
-0.64278760968654	0.766044443118977\\
-0.59290792905464	0.805270257531059\\
-0.540640817455597	0.841253532831181\\
-0.486196736100469	0.873849377069785\\
-0.429794912089172	0.902926538286621\\
-0.371662455660328	0.928367933016072\\
-0.312033445698487	0.950071117740945\\
-0.251147987181079	0.967948701396356\\
-0.189251244360411	0.981928697262707\\
-0.12659245357375	0.991954812830795\\
-0.0634239196565654	0.997986676471884\\
-2.44929359829471e-16	1\\
};
\addplot [color=black, dashed, forget plot]
  table[row sep=crcr]{%
1	0\\
0.985749400778687	0.03129154540054\\
0.970722720321987	0.0616912442560429\\
0.954958925509202	0.0911875211329771\\
0.93849679013146	0.119770184888548\\
0.921374854326737	0.147430402901699\\
0.903631385399313	0.174160674401991\\
0.885304340032149	0.19995480295255\\
0.866431327898817	0.224807868142708\\
0.847049576679755	0.248716196545405\\
0.827195898485859	0.27167733199377\\
0.806906657690643	0.293690005230622\\
0.786217740170526	0.3147541029839\\
0.765164523951142	0.334870636520268\\
0.743781851255967	0.354041709728332\\
0.722104001951983	0.372270486782039\\
0.700164668385638	0.389561159433977\\
0.677996931600836	0.405918913987337\\
0.65563323892934	0.421349897994367\\
0.633105382942583	0.435861186728169\\
0.610444481752575	0.449460749473654\\
0.587680960648341	0.462157415682464\\
0.564844535053122	0.473960841035577\\
0.541964194786388	0.484881473456244\\
0.519068189613649	0.494930519114805\\
0.496184016065939	0.504119908465775\\
0.473338405509889	0.512462262356494\\
0.450557313448321	0.519970858245437\\
0.427865910030392	0.526659596567132\\
0.405288571749477	0.532542967279455\\
0.382848874306133	0.537636016627846\\
0.36056958661279	0.541954314159836\\
0.338472665916026	0.545513920022031\\
0.316579254011691	0.548331352570503\\
0.294909674527489	0.550423556324316\\
0.273483431247061	0.551807870290704\\
0.252319207449113	0.552501996689188\\
0.231434866234628	0.552523970100718\\
0.210847451814787	0.551892127066692\\
0.190573191731831	0.550625076161518\\
0.17062749998475	0.548741668561138\\
0.151024981031384	0.546260969128792\\
0.131779434638263	0.543202228038051\\
0.112903861549295	0.539584852951998\\
0.0944104699442263	0.535428381776254\\
0.0763106826576658	0.53075245600239\\
0.0586151451293598	0.525576794657101\\
0.0413337340563391	0.519921168871396\\
0.0244755667175409	0.513805377082931\\
0.00804901094150263	0.507249220883512\\
-0.00793830431221784	0.500272481522684\\
-0.0234794777871754	0.492894897077263\\
-0.0385683242088924	0.485136140295627\\
-0.0531993620374572	0.477015797124494\\
-0.0673678007227277	0.468553345924963\\
-0.0810695274907928	0.459768137383532\\
-0.0943010936901188	0.450679375122871\\
-0.107059700725554	0.441306097016158\\
-0.119343185608077	0.431667157207837\\
-0.131150006147872	0.421781208842765\\
-0.142479225817984	0.411666687504809\\
-0.153330498315436	0.401341795365095\\
-0.16370405184634	0.39082448603926\\
-0.173600673161113	0.380132450152237\\
-0.183021691365501	0.369283101608317\\
-0.191968961532673	0.358293564563446\\
-0.200444848141198	0.347180661095983\\
-0.208452208363235	0.335960899571399\\
-0.215994375226798	0.324650463695734\\
-0.223075140675424	0.313265202251922\\
-0.229698738548096	0.30182061951247\\
-0.2358698275017	0.290331866321345\\
-0.241593473897791	0.278813731837337\\
-0.246875134674871	0.267280635930577\\
-0.251720640226819	0.25574662222337\\
-0.256136177307551	0.24422535176595\\
-0.260128271981409	0.232730097337309\\
-0.263703772638172	0.221273738360748\\
-0.26686983309102	0.209868756423366\\
-0.269633895775158	0.198527231388299\\
-0.272003675064216	0.187260838088095\\
-0.273987140720928	0.176080843587269\\
-0.275592501497997	0.164998105001723\\
-0.276828188904415	0.154023067862394\\
-0.277702841151923	0.143165765010198\\
-0.278225287295666	0.132435816009061\\
-0.278404531582486	0.121842427063578\\
-0.278249738019679	0.111394391427613\\
-0.277770215176448	0.101100090289945\\
-0.276975401229657	0.090967494122873\\
-0.27587484926489	0.0810041644795503\\
-0.274478212843221	0.0712172562256555\\
-0.272795231843497	0.0616135201908994\\
-0.270835718589351	0.0521993062257631\\
-0.268609544269558	0.0429805666487825\\
-0.266126625659784	0.0339628600696366\\
-0.263396912153188	0.025151355573249\\
-0.260430373106767	0.0165508372500984\\
-0.257236985509794	0.00816570905792015\\
-0.253826721980109	3.10848082611005e-17\\
-0.253826721980109	-3.10848082611005e-17\\
-0.257236985509794	-0.00816570905792015\\
-0.260430373106767	-0.0165508372500984\\
-0.263396912153188	-0.025151355573249\\
-0.266126625659784	-0.0339628600696366\\
-0.268609544269558	-0.0429805666487825\\
-0.270835718589351	-0.0521993062257631\\
-0.272795231843497	-0.0616135201908994\\
-0.274478212843221	-0.0712172562256555\\
-0.27587484926489	-0.0810041644795503\\
-0.276975401229657	-0.090967494122873\\
-0.277770215176448	-0.101100090289945\\
-0.278249738019679	-0.111394391427613\\
-0.278404531582486	-0.121842427063578\\
-0.278225287295666	-0.132435816009061\\
-0.277702841151923	-0.143165765010198\\
-0.276828188904415	-0.154023067862394\\
-0.275592501497997	-0.164998105001723\\
-0.273987140720928	-0.176080843587269\\
-0.272003675064216	-0.187260838088095\\
-0.269633895775158	-0.198527231388299\\
-0.26686983309102	-0.209868756423366\\
-0.263703772638172	-0.221273738360748\\
-0.260128271981409	-0.232730097337309\\
-0.256136177307551	-0.24422535176595\\
-0.251720640226819	-0.25574662222337\\
-0.246875134674871	-0.267280635930577\\
-0.241593473897791	-0.278813731837337\\
-0.2358698275017	-0.290331866321345\\
-0.229698738548096	-0.30182061951247\\
-0.223075140675424	-0.313265202251922\\
-0.215994375226798	-0.324650463695734\\
-0.208452208363235	-0.335960899571399\\
-0.200444848141198	-0.347180661095983\\
-0.191968961532673	-0.358293564563446\\
-0.183021691365501	-0.369283101608317\\
-0.173600673161113	-0.380132450152237\\
-0.16370405184634	-0.39082448603926\\
-0.153330498315436	-0.401341795365095\\
-0.142479225817984	-0.411666687504809\\
-0.131150006147872	-0.421781208842765\\
-0.119343185608077	-0.431667157207837\\
-0.107059700725554	-0.441306097016158\\
-0.0943010936901188	-0.450679375122871\\
-0.0810695274907928	-0.459768137383532\\
-0.0673678007227277	-0.468553345924963\\
-0.0531993620374572	-0.477015797124494\\
-0.0385683242088924	-0.485136140295627\\
-0.0234794777871754	-0.492894897077263\\
-0.00793830431221784	-0.500272481522684\\
0.00804901094150263	-0.507249220883512\\
0.0244755667175409	-0.513805377082931\\
0.0413337340563391	-0.519921168871396\\
0.0586151451293598	-0.525576794657101\\
0.0763106826576658	-0.53075245600239\\
0.0944104699442263	-0.535428381776254\\
0.112903861549295	-0.539584852951998\\
0.131779434638263	-0.543202228038051\\
0.151024981031384	-0.546260969128792\\
0.17062749998475	-0.548741668561138\\
0.190573191731831	-0.550625076161518\\
0.210847451814787	-0.551892127066692\\
0.231434866234628	-0.552523970100718\\
0.252319207449113	-0.552501996689188\\
0.273483431247061	-0.551807870290704\\
0.294909674527489	-0.550423556324316\\
0.316579254011691	-0.548331352570503\\
0.338472665916026	-0.545513920022031\\
0.36056958661279	-0.541954314159836\\
0.382848874306133	-0.537636016627846\\
0.405288571749477	-0.532542967279455\\
0.427865910030392	-0.526659596567132\\
0.450557313448321	-0.519970858245437\\
0.473338405509889	-0.512462262356494\\
0.496184016065939	-0.504119908465775\\
0.519068189613649	-0.494930519114805\\
0.541964194786388	-0.484881473456244\\
0.564844535053122	-0.473960841035577\\
0.587680960648341	-0.462157415682464\\
0.610444481752575	-0.449460749473654\\
0.633105382942583	-0.435861186728169\\
0.65563323892934	-0.421349897994367\\
0.677996931600836	-0.405918913987337\\
0.700164668385638	-0.389561159433977\\
0.722104001951983	-0.372270486782039\\
0.743781851255967	-0.354041709728332\\
0.765164523951142	-0.334870636520268\\
0.786217740170526	-0.3147541029839\\
0.806906657690643	-0.293690005230622\\
0.827195898485859	-0.27167733199377\\
0.847049576679755	-0.248716196545405\\
0.866431327898817	-0.224807868142708\\
0.885304340032149	-0.19995480295255\\
0.903631385399313	-0.174160674401991\\
0.921374854326737	-0.147430402901699\\
0.93849679013146	-0.119770184888548\\
0.954958925509202	-0.0911875211329771\\
0.970722720321987	-0.0616912442560429\\
0.985749400778687	-0.03129154540054\\
1	0\\
nan	0\\
};

\addplot [color=black, forget plot]
  table[row sep=crcr]{%
  0.241172754840028 0.552596202525959\\
  -0.253826721980109 3.10848082611005e-17\\
  0.241172754840028	-0.552596202525959\\
};
\addplot [color=black, forget plot]
  table[row sep=crcr]{%
0.241172754840028	0.552596202525959\\
1	0\\
0.241172754840028	-0.552596202525959\\
};

\coordinate (A) at (0.241172754840028, 0.552596202525959);
\coordinate (B) at (-0.253826721980109, 3.10848082611005e-17);
\coordinate (C) at (1, 0);
\pic [draw, angle radius = 3mm] {angle=C--B--A};
\pic [draw, angle radius = 3mm] {angle=A--C--B};
\end{axis}

\draw (3.1,4.4) node[scale = 0.65, anchor=south] {\small $V_i$};
\draw (8.45,4.4) node[scale = 0.65, anchor=south] {\small $V_o$};
\draw (5.4,6.5) node[scale = 0.65, anchor=south] {\small $V$};
\draw (5.4,2.3) node[scale = 0.65, anchor=north] {\small $\bar{V}$};

\end{tikzpicture}%
\end{figure}

\begin{figure}
  \centering
  % This file was created by matlab2tikz.
%
%The latest updates can be retrieved from
%  http://www.mathworks.com/matlabcentral/fileexchange/22022-matlab2tikz-matlab2tikz
%where you can also make suggestions and rate matlab2tikz.
%
%
\begin{tikzpicture}

\begin{axis}[%
  axis lines=center,
  width=3.5in,
  height=3.5in,
  scale only axis,
  xmin=-1.2,
  xmax=1.2,
  ymin=-1.2,
  ymax=1.2,
  xtick={1},
  ytick=\empty,
  x label style={anchor=north},
  xticklabel style={anchor=south west},
  xlabel={$\pmb{X}$},
  ylabel={$\pmb{jY}$}
]
\addplot [color=black, forget plot]
  table[row sep=crcr]{%
0	1\\
0.0634239196565645	0.997986676471884\\
0.126592453573749	0.991954812830795\\
0.18925124436041	0.981928697262707\\
0.251147987181079	0.967948701396356\\
0.312033445698487	0.950071117740945\\
0.371662455660328	0.928367933016073\\
0.429794912089172	0.902926538286621\\
0.486196736100469	0.873849377069785\\
0.540640817455598	0.841253532831181\\
0.59290792905464	0.805270257531059\\
0.642787609686539	0.766044443118978\\
0.690079011482112	0.72373403810507\\
0.734591708657533	0.678509411557132\\
0.776146464291757	0.630552667084523\\
0.814575952050336	0.580056909571198\\
0.849725429949514	0.527225467610502\\
0.881453363447582	0.472271074772683\\
0.909631995354518	0.415415013001886\\
0.934147860265107	0.356886221591872\\
0.954902241444074	0.296920375328275\\
0.971811568323542	0.235758935509427\\
0.984807753012208	0.173648177666931\\
0.993838464461254	0.110838199901011\\
0.998867339183008	0.0475819158237424\\
0.999874127673875	-0.015865963834808\\
0.996854775951942	-0.0792499568567885\\
0.989821441880933	-0.142314838273285\\
0.978802446214779	-0.204806668065191\\
0.963842158559942	-0.266473813690035\\
0.945000818714669	-0.327067963317421\\
0.922354294104581	-0.386345125693128\\
0.895993774291336	-0.444066612605774\\
0.866025403784439	-0.5\\
0.832569854634771	-0.55392006386611\\
0.795761840530832	-0.605609687137666\\
0.755749574354258	-0.654860733945285\\
0.712694171378863	-0.701474887706321\\
0.666769000516292	-0.745264449675755\\
0.618158986220605	-0.786053094742787\\
0.567059863862771	-0.823676581429833\\
0.513677391573407	-0.857983413234977\\
0.458226521727411	-0.888835448654923\\
0.400930535406614	-0.916108457432069\\
0.342020143325669	-0.939692620785908\\
0.28173255684143	-0.959492973614497\\
0.220310532786541	-0.975429786885407\\
0.15800139597335	-0.987438888676394\\
0.0950560433041829	-0.995471922573085\\
0.0317279334980681	-0.999496542383185\\
-0.0317279334980679	-0.999496542383185\\
-0.0950560433041826	-0.995471922573085\\
-0.15800139597335	-0.987438888676394\\
-0.220310532786541	-0.975429786885407\\
-0.281732556841429	-0.959492973614497\\
-0.342020143325669	-0.939692620785908\\
-0.400930535406613	-0.91610845743207\\
-0.45822652172741	-0.888835448654924\\
-0.513677391573406	-0.857983413234977\\
-0.567059863862771	-0.823676581429833\\
-0.618158986220605	-0.786053094742788\\
-0.666769000516292	-0.745264449675755\\
-0.712694171378863	-0.701474887706322\\
-0.755749574354258	-0.654860733945285\\
-0.795761840530832	-0.605609687137667\\
-0.832569854634771	-0.55392006386611\\
-0.866025403784438	-0.5\\
-0.895993774291336	-0.444066612605774\\
-0.922354294104581	-0.386345125693129\\
-0.945000818714668	-0.327067963317422\\
-0.963842158559942	-0.266473813690035\\
-0.978802446214779	-0.204806668065191\\
-0.989821441880933	-0.142314838273285\\
-0.996854775951942	-0.0792499568567888\\
-0.999874127673875	-0.0158659638348076\\
-0.998867339183008	0.0475819158237424\\
-0.993838464461254	0.110838199901011\\
-0.984807753012208	0.17364817766693\\
-0.971811568323542	0.235758935509427\\
-0.954902241444074	0.296920375328275\\
-0.934147860265107	0.356886221591872\\
-0.909631995354519	0.415415013001886\\
-0.881453363447582	0.472271074772682\\
-0.849725429949514	0.527225467610502\\
-0.814575952050336	0.580056909571198\\
-0.776146464291757	0.630552667084522\\
-0.734591708657534	0.678509411557132\\
-0.690079011482113	0.723734038105069\\
-0.64278760968654	0.766044443118977\\
-0.59290792905464	0.805270257531059\\
-0.540640817455597	0.841253532831181\\
-0.486196736100469	0.873849377069785\\
-0.429794912089172	0.902926538286621\\
-0.371662455660328	0.928367933016072\\
-0.312033445698487	0.950071117740945\\
-0.251147987181079	0.967948701396356\\
-0.189251244360411	0.981928697262707\\
-0.12659245357375	0.991954812830795\\
-0.0634239196565654	0.997986676471884\\
-2.44929359829471e-16	1\\
};
\addplot [color=white!15!black, dashed, forget plot]
  table[row sep=crcr]{%
0.54030230586814	0.841470984807897\\
0.523839075837643	0.814928316186227\\
0.508687320595882	0.788736271322035\\
0.494768449644869	0.762935534189694\\
0.482005752907937	0.7375600143713\\
0.47032474799982	0.712637387522045\\
0.459653461283272	0.688189626551107\\
0.44992264841671	0.664233518320049\\
0.441065960151734	0.640781161632712\\
0.433020059085366	0.617840443185288\\
0.425724692928414	0.595415488957203\\
0.419122729635893	0.573507089249993\\
0.413160159474338	0.55211309622246\\
0.407786068788665	0.53122879332768\\
0.402952589891059	0.510847236534364\\
0.398614831137396	0.490959567616075\\
0.394730790892751	0.471555300122364\\
0.391261258724557	0.452622578911893\\
0.388169706806669	0.434148414335319\\
0.385422174175058	0.416118892311548\\
0.382987146150303	0.39851936165114\\
0.380835430936167	0.381334600051309\\
0.378940035119482	0.36454896022382\\
0.377276039535394	0.34814649762556\\
0.375820476724226	0.332111081246612\\
0.374552210991787	0.316426488876725\\
0.373451821893269	0.301076488222204\\
0.372501491791189	0.286044905184962\\
0.371684897988954	0.271315680546931\\
0.370987109812263	0.256872916228713\\
0.370394490899359	0.242700912213813\\
0.369894606866539	0.22878419515056\\
0.369476138435956	0.215107539564813\\
0.369128800046965	0.201655982538708\\
0.368843263918867	0.188414832635262\\
0.368611089490272	0.175369673776221\\
0.368424658127448	0.162506364711798\\
0.368277112969447	0.149811034656222\\
0.368162303760724	0.137270075602611\\
0.368074736511088	0.124870131774795\\
0.368009527817544	0.112598086622262\\
0.367962363681836	0.100441047717563\\
0.367929462660766	0.0883863298730892\\
0.367907543192914	0.0764214367560594\\
0.367893794954673	0.0645340412467305\\
0.367885854110147	0.0527119647551002\\
0.367881782332874	0.0409431556855508\\
0.367880049492378	0.0292156672168579\\
0.367879519914666	0.0175176345466192\\
0.367879442142994	0.00583725173430473\\
0.367879442142994	-0.00583725173430464\\
0.367879519914666	-0.0175176345466191\\
0.367880049492378	-0.0292156672168578\\
0.367881782332874	-0.0409431556855508\\
0.367885854110147	-0.0527119647551001\\
0.367893794954673	-0.0645340412467302\\
0.367907543192914	-0.0764214367560592\\
0.367929462660766	-0.0883863298730889\\
0.367962363681836	-0.100441047717563\\
0.368009527817544	-0.112598086622262\\
0.368074736511088	-0.124870131774795\\
0.368162303760724	-0.137270075602611\\
0.368277112969447	-0.149811034656221\\
0.368424658127448	-0.162506364711798\\
0.368611089490272	-0.175369673776221\\
0.368843263918867	-0.188414832635262\\
0.369128800046965	-0.201655982538708\\
0.369476138435956	-0.215107539564813\\
0.369894606866539	-0.22878419515056\\
0.370394490899359	-0.242700912213813\\
0.370987109812263	-0.256872916228713\\
0.371684897988954	-0.271315680546931\\
0.372501491791189	-0.286044905184962\\
0.373451821893269	-0.301076488222203\\
0.374552210991787	-0.316426488876726\\
0.375820476724225	-0.332111081246612\\
0.377276039535394	-0.34814649762556\\
0.378940035119482	-0.36454896022382\\
0.380835430936167	-0.381334600051309\\
0.382987146150303	-0.39851936165114\\
0.385422174175058	-0.416118892311548\\
0.388169706806669	-0.43414841433532\\
0.391261258724557	-0.452622578911893\\
0.394730790892751	-0.471555300122364\\
0.398614831137396	-0.490959567616075\\
0.402952589891059	-0.510847236534364\\
0.407786068788665	-0.53122879332768\\
0.413160159474337	-0.552113096222459\\
0.419122729635893	-0.573507089249993\\
0.425724692928414	-0.595415488957203\\
0.433020059085366	-0.617840443185288\\
0.441065960151734	-0.640781161632711\\
0.44992264841671	-0.664233518320049\\
0.459653461283271	-0.688189626551107\\
0.47032474799982	-0.712637387522044\\
0.482005752907937	-0.7375600143713\\
0.494768449644869	-0.762935534189693\\
0.508687320595882	-0.788736271322035\\
0.523839075837643	-0.814928316186226\\
0.54030230586814	-0.841470984807896\\
nan	0\\
};

\addplot [color=blue, forget plot]
  table[row sep=crcr]{%
0.54030230586814	0.841470984807897\\
0.367879441171442	0\\
0.54030230586814	-0.841470984807897\\
};
\end{axis}
\end{tikzpicture}%
\end{figure}

\begin{algorithm}[hbt!]
  \caption{Aproximação cônica da taxa de amortecimento}\label{alg:AproximacaoConicaZeta}
  \begin{algorithmic}[1]
    \Require $\zeta$, $T_s$
    \Ensure $K$
    \State $Z_o \gets $ $z(\zeta,0)$
    \State $Z_i \gets $ $z\left(\zeta,\dfrac{\pi}{T_s\sqrt(1-\zeta^2)}\right)$
    \State $Z \gets z(\zeta,\omega_n)$, onde a área do triângulo formado é a maior possível
    \State $F \gets P \succ 0$
    \State $F \gets F \cap \eqref{eq:LMIESetorConicoEsquerdo}$, com $a = Z_o$ e $\theta = ang(Z,Z_i)$ \Comment{Setor cônico esquerdo}
    \State $F \gets F \cap \eqref{eq:LMIESetorConicoDireito}$, com $a = Z_i$ e $\theta = ang(Z,Z_o)$ \Comment{Setor cônico direito}
    \State $F \gets F \cap \eqref{eq:LMIRightBounded}$, com $a = Z_i$  \Comment{Reta vertical}
    \State Verificar se o problema é factível
    \State $K \gets ZP^{-1}$
  \end{algorithmic}
\end{algorithm}

\begin{algorithm}[hbt!]
  \caption{Aproximação cônica da curva $N_y$}\label{alg:AproximacaoConicaNy}
  \begin{algorithmic}[1]
    \Require $\omega_n$
    \Ensure $K$
    \State $N_o \gets $ $z(0,\omega_n)$
    \State $N_i \gets $ $z(1,\omega_n)$
    \State $F \gets P \succ 0$
    \State $F \gets F \cap \eqref{eq:LMIESetorConicoDireito}$, com $a = N_i$ e $\theta = ang(N_i,N_o)$ \Comment{Setor cônico direito}
    \State $F \gets F \cap \eqref{eq:LMIRightBounded}$, com $a = N_i$ \Comment{Reta vertical}
    \State Verificar se o problema é factível
    \State $K \gets ZP^{-1}$
  \end{algorithmic}
\end{algorithm}